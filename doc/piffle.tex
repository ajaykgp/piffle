% Report on Piffle, A Packet Filter Language
% by Jaap Weel

\documentclass[a4paper,12pt]{scrreprt}
\usepackage{amsmath}
\usepackage{amssymb}
\usepackage{fancyvrb}
\titlehead{Vrije Universiteit Amsterdam\\ MPhil Parallel and
  Distributed Computer Systems \\ Compiler
  Construction Practical}
\title{Piffle\\ A Packet Filter Language}
\author{Jaap Weel}
\date{Version 
}

\begin{document}
\maketitle

\tableofcontents

\chapter{Notices}
\section{Latest version of this document}

For up-to-date information on Piffle, and the latest version of this
document, be sure to visit the following web site:

 \textit{http://code.google.com/p/piffle/wiki/PiffleWiki}

\section{Copyright notice}

Copyright (c) 2007, Jaap Weel. This work is licensed under
the Creative Commons Attribution--Share-Alike 3.0 License. To view a
copy of this license, visit

\textit{http://creativecommons.org/licenses/by-sa/3.0/} 

or send a
letter to Creative Commons, 171 Second Street, Suite 300, San
Francisco, California, 94105, USA.



\chapter{Introduction}

Piffle is a pronounceable form of PFL, that is, Packet Filter
Language. (``Piffle'' is also an obscure word for ``chatter''.) Piffle
is a very simple programming language for writing network packet
filters, with one special feature: the compiler can enforce limits on
the processor time and memory used for processing each packet of
input. This is accomplished by passing arguments to the --C and --M
command line options. Memory bounds are in words, more or less, and
CPU bounds are in arbitrary units.

In this document, I will first consider in general the problem of
checking arbitrary programs for transgression of such resource bounds,
and notice that it is very hard. I will conclude that in Piffle we must
punt and restrict the class of programs that can be used as packet
filters, which makes the problem comparatively easy.

After that, I will give the definition of the Piffle language, and
some instructions for how to use the Piffle compiler.


\chapter{Bounded packet filters}

An important requirement in designing network packet processing code
is that the amount of time and memory spent processing each packet be
bounded.  We do not want the entire network connection grinding to a
halt because of an unforeseen infinite loop in a packet filter. There
are several ways to prevent such havoc. In this chapter, I~will
discuss each of them, and explain which one I chose and why. 

\section{Dynamic bounds on packet filters}

We can check bounds at runtime. For instance, we can run the packet
filter on each packet for a specified quota of clock ticks and drop
the packet if processing has not finished by the time the quota runs
out.

One problem with this runtime approach is that a bug in the packet
filters can remain hidden for a long time until some unusual packet is
received that triggers the bug.

Another problem is that it is not clear exactly what to do when the
bound is exceeded: Should we drop the packet? Should we pass it along
unaltered? We would like to be able to give the packet filter author
the opportunity to decide what to do with packets that take too long
to analyze. But if we allow that sort of flexibility within the
dynamic checking framework, we will inevitably have to allow a chunk
of cleanup code to deal with ``slow'' packets, which in turn can only
run for a restricted amount of time. If \textit{that} time is
exceeded, do we need a third program to deal with slow packet filters
that have slow cleanup code?

\section{Static bounds on arbitrary packet filters}

Given the problems with checking bounds at run time, we might want to
check at compile time. To do that, we need a language in which we can
express exactly those packet filter programs that execute within given
time and space bounds.

Determining whether a given program runs within these bounds sounds
like it ought to be undecidable, but it is not. After all, we are not
trying to evaluate whether a given packet filter will terminate
\textit{at all, in the end}; we merely want it to terminate within a
given number of steps. (The space bound problem is analogous.)

The problem of determining whether a given Turing machine will
terminate within a certain number of steps is, in fact, decidable.  In
a limited number $N$ of steps the machine can read only a limited
amount $N$ of its tape, which means we could check whether the time
bound by exhaustively running the machine for at most $N$ steps on all
tapes of length at most $N$.

It should be obvious that this procedure, while quite satisfactory as
a constructive proof of decidability, is impractical, because the
process of bounds checking itself would take lots of time (in fact,
exponential in $N$).  What we need is a bound on the process of bounds
checking.

\section{Proof carrying code}

One way to design a language that would let us check bounds at compile
time would be to annotate the program with a proof of the statement
that it will terminate within a given number of steps.  

Proofs could be provided either as annotations or as part of a type
system.\footnote{Crary and Weirich. ``Resource Bound Certification.''
27th \textsc{popl}, p.184.}  The use of proof carrying code would
allow all packet filters to be written that can be proven to execute
within bounds, and it would allow for the bounds checking problem
itself to be tractable.  Depending on the details of the formal system
used, proof (or type) checking can be polynomial (and often linear) in
the length of the proof.  Therefore, bound checking of annotated code
complies with the rule of thumb that compilers should use only
algorithms that are at most polynomial in the length of the program.

The practical liabilities of proof carrying code are well
known. Programmers are unfamiliar with proof systems, and those who
know their way around them could still have trouble coming up with the
(possibly lengthy) proofs in individual cases. Therefore, we will not
pursue proof carrying code any further now, and turn instead to a
simpler solution.

\section{Bounds on restricted packet filters}

If we do not want to get bogged down in the depths of formal type
theory, but still design a language that would let us check bounds at
compile time, we arrive at what is, in fact, the traditional solution
to the problem of bounded packet filters. We will construct a
programming language out of familiar abstractions, and that language
will allow us to express a \textit{large but arbitrary} subset of the
set of all packet filters that execute in bounded time. 

Such a language is not Turing complete, but if it is designed
intelligently, it can cover a large fraction of the problems that
people may want to solve using packet filters.

An example of a packet filter language based on familiar programming
abstractions is the traditional BSD packet filter
\textit{bpf}.\footnote{McCanne and Van Jacobson. ``The BSD packet
filter.'' \textsc{usenix} 1993.} Its packet filters are written in a simple
machine code with a little twist: the virtual machine allows jumps
forward, but not backward. That way, the time spent executing any
given packet filter program is at most proportional to the length of
the program.

For Piffle, we take a similar approach. Piffle is not based on machine
language, but on block structured, procedural programming languages
such as C and Pascal. Just as the \textit{bpf} language omits backward
jumps from machine code, Piffle omits unbounded looping constructs and
recursive procedure calls from block structured, procedural
code. Piffle does provide some bounded looping constructs, and
procedure calls are allowed as long as the call graph contains no
cycles.


\chapter{Language definition}

%% Definitions used for BNF
\def\code#1{\texttt{#1}}
\def\id#1{\,\text{\it #1}\,}
\def\lit#1{\;\code{\fbox{\raisebox{-0.1em}{\rule{0em}{0.8em}}#1}}\;}
\def\grp#1{\;(#1)\,}
\def\is{&=&}
\def\star{^*}
\def\plus{^+}
\def\opt{^?}
\def\bar{\,|\,}
\def\bbar{\\&\bar&}
\def\clause#1{\begin{eqnarray*}#1\end{eqnarray*}}


The Piffle language is defined as a set of sentences, called source
files, well formed according to the Piffle syntax. To describe the
syntax, I will use a modified version of the Backus Normal Form (also
known as Backus-Naur Formalism or BNF), using $x^?$ to indicate that
$x$ may appear either once or not at all, $x^+$ to indicate that $x$
may appear at least once, and $x^*$ to indicate that $x$ may appear
any number of times, including not at all.

I will not provide a formal definition of the semantics of Piffle. In
general, when I do not specify the meaning or type of some Piffle
construct, it can be assumed that it is analogous to the semantics of
its closest analog in C.

\section{Vocabulary and representation}

A well formed Piffle program is composed of tokens, which are
identifiers, literals, and reserved tokens. Each token is a sequence
of characters, and may not contain whitespace unless this is
explicitly specified. Any amount of whitespace may be inserted between
tokens, and indeed must be inserted between any two tokens that would
make a valid token when put together.

Whitespace is any sequence of blanks, tabs, vertical tabs, form feeds,
line feeds, and carriage returns (in C notation, any character in the
string \verb." \v\f\t\r\n".). Comments may also appear wherever
whitespace can appear. (Unlike in C, they may \textit{not} appear
inside of a token.) In-line comments may be inserted between any
two tokens. They are delimited by \code{/*} and \code{*/} and may be
nested. End-of-line comments may also be inserted between any two
tokens, and they run from \code{//} to the next line end, or the end of
the file, whichever comes first.

Identifiers are sequences of letters, digits, and underscores. The
first character may not be a digit. 
\clause{%
  \id{ident} \is \grp{ \id{letter} \bar \lit\_ }
                 \grp{ \id{letter} \bar \lit\_ \bar \id{digit} }\star\\
  \id{letter} \is \lit A \bar \cdots \bar \lit Z \bar
                  \lit a \bar \cdots \bar \lit z \\
  \id{digit} \is \lit 0 \bar \cdots \bar \lit 9
}

The literals available in Piffle, are numbers, the \code{unit}
literal, which is the canonical value of type \code{void},
\code{true}, which means the same as \code{1 :~bool}, and
\code{false}, which means the same as \code{0 :~bool}. The only
numbers available are integers. They can be represented as decimal,
octal, or hexidecimal, the same way as in Haskell 98. Note that this
differs from C syntax in that leading 0s are simply ignored, and the
prefix for octal is 0o or 0O rather than simply 0.

\clause{%
  \id{literal} \is \lit{unit} \bar \lit{true} \bar \lit{false}
                    \bar \id{integer}\\
  \id{integer} \is \id{digit}\plus \bar
		   \grp{\lit{0o} \bar \lit{0O}}\id{octit}\plus \bar
		   \grp{\lit{0x} \bar \lit{0X}} \id{hexit}\plus\\
  \id{hexit} \is \lit 0 \bar \cdots \bar \lit 9 \bar \lit A \bar \cdots
		 \bar \lit F \bar \lit a \bar \cdots \bar \lit f \\ 
  \id{octit} \is \lit 0 \bar \cdots \bar \lit 7
}

\section{Declarations}

Piffle allows no recursion, and all functions must be defined before
they are used. Therefore, there is no need for forward
declarations. If the return type of a function is omitted, \code{void}
is implied. The meaning of duplicate definitions is undefined.

\clause{%
  \id{declaration} \is \id{vardec} \bar \id{fundec} \\
  \id{vardec} \is \lit{var} \id{ident} \lit : \id{type}\\
  \id{fundec} \is \lit{fun} \id{ident} \lit ( \id{argl} \opt\lit )
                  \grp{\lit : \id{type}}\opt \lit = 
                  \id{expression}  \\
  \id{argl} \is \id{ident} \lit : \id{type} 
                \grp{\lit , \id{ident} \lit : \id{type}}\star 
}

\section{Types}

A type in Piffle is \code{void}, an atomic type, or an array. Atomic
types come in signed and unsigned, and have 8, 16, or 32 bits.  Each
array type encompasses only arrays of a specific length; you can
declare an ``array of 1024 integers'', but not an ``array of
integers''. This is an important feature, because it is what enables
the calculation of resource limits. It is at present \textit{not}
actually used for anything else; in particular, there are no bounds
checks.

\clause{%
  \id{type} \is \lit{void}\bar 
                \id{atomic}\bar
                \id{atomic} \lit [ \id{integer} \lit ] \\
  \id{atomic} \is \lit{bool} \bar
              \lit{u8} \bar \lit{u16} \bar \lit{u32} 
              \bar \lit{s8} \bar \lit{s16} \bar \lit{s32}
}

Whenever a value of type of void is expected, a value of any other
type will do as well.

\section{Expressions}

Unlike in C or Pascal, in Piffle there is no syntactic distinction
between expressions and statements; every expression can be used as a
statement and vice versa.

\clause{%
  \id{expression} \is
   \id{expression2} \grp{\id{binop} \id{expression2}}\star\\
  \id{expression2} \is
    \id{unop}\star \id{expression3} 
    \grp{\lit{[}\id{expression}\lit{]}}\opt
    \grp{\lit: \id{atomic}}\opt\\
  \id{expression3} 
   \is   \id{literal} 
   \bbar \id{ident} 
           \grp{\lit{(} 
           \grp{ \id{expression}
             \grp{\lit,\id{expression}}\star }\opt
           \lit{)}} \opt
   \bbar \lit( \id{expression} \lit) 
   \bbar \lit\{ \grp{\id{declaration}\lit;}\star
          \grp{\id{expression}\lit;}\star \lit\} 
   \bbar \lit{if} \id{expression}
         \lit{then} \id{expression}
	 \grp{\lit{else} \id{expression}}\opt
   \bbar \lit{for} \id{ident} \lit{in} \id{expression}
         \grp{\lit{while} \id{expression}}\opt \lit{do}
	 \id{expression}
   \bbar \lit{for} \id{ident} \lit{from} \id{integer} \lit{to} 
         \id{integer} \\&& \grp{\lit{while} \id{expression}}\opt 
	 \lit{do} \id{expression}
}

Cast expressions (like \code{x : u32}) allow the use of a value of one
atomic type with another atomic type. The exact meaning of such
conversions is undefined but expected to correspond to the behavior of
the C compiler on the platform in use.

Blocks have the value of the last expression in them (which is why
there is no \code{return} statement); if a block contains no
expressions, \code{unit} is implied. 

If-then-else statements have values, and correspond more closely to
the C ternary operator (\code{?:}) than to the C if-else statement. If
an \code{else} clause is omitted, \code{unit} is implied.

Loops in Piffle are somewhat peculiar, in order to accommodate
resource bound checking. There are two types of loops: one will
iterate with a variable sequentially bound to each number in a range
(which includes both bounds), and the other will iterate with a
variable sequentially bound to each element in an array. Both allow
the use of a \code{while} clause, which will be evaluated before each
iteration; if it evaluates to 0, the loop is aborted immediately. The
iterator variable is bound within the while clause, so it is perfectly
valid to say something like \code{for i from 0 to 10 while
is\_useful(i) do ...}.


\section{Operators}

The operators are given by
\clause \bar \lit{+} \bar \lit{-} \bar
    \lit{<<} \bar \lit{>>} \bar \lit{<}\bbar \lit{>} \bar \lit{<=}
    \bar \lit{>=} \bar \lit{==} \bar \lit{!=} \bar \lit{\&} \bar
    \lit{\^{}} \bbar \lit{|}\bar \lit{\&\&} \bar \lit{||} \bar
    \lit{=} \bar \lit{+=} \bar \lit{-=} \bar \lit{*=} \bbar
    \lit{\%=}\bar \lit{/=} \bar \lit{|=} \bar \lit{\&=} \bar
    \lit{\^{}=} \bar \lit{>>=} \bar \lit{<<=} \\
  \id{unop} \is
    \lit{+} \bar \lit{-} \bar \lit{\~{}} \bar \lit{!}
}

The grammar as specified so far correctly and completely specifies the
language, but it is misleading and ambiguous, because it does not
indicate the meaning of certain compound expressions. In fact, the
actual Piffle parser will respect the standard C precedence rules, as
summarized in the following table.

\def\x{\phantom{x}}
\vspace{1em}
\begin{tabular}{ll}
\hline\hline
Operators (ordered from tight to loose) & Associativity\\
\hline
\verb.*. \x \verb./. \x \verb.%. & left-to-right \\
\verb.+. \x \verb.-. & left-to-right \\
\verb.<<. \x \verb.>>. & left-to-right \\
\verb.<. \x \verb.>. \x \verb.<=. \x \verb.>=. & left-to-right \\
\verb.==. \x \verb.!=. & left-to-right \\
\verb.&. & left-to-right \\
\verb.^. & left-to-right \\
\verb.|. & left-to-right \\
\verb.&&. & left-to-right \\
\verb.||. & left-to-right \\
\verb.=. \x \verb.+=. \x \verb.-=. \x \verb.*=. \x \verb.%=. \x
\verb./=. \x \verb.|=. \x \verb.&=. \x \verb.^=. \x
\verb.>>=. \x \verb.<<=. & right-to-left\\
\hline\hline
\end{tabular}
\vspace{1em}

The keywords have the loosest possible preference, and thus the
expressions at the end of \code{if} and \code{for} expressions extend
to the right as far as possible.

Even though any expression may be used, according to the grammar, as
the left-hand side of an assignment, assignment is only a meaningful
operation when the left-hand side is an ``lvalue'' of the form
$\id{ident}\grp{\lit{[}\id{expression}\lit{]}}\opt$.

\section{The types of expressions}

The binary operators \code{*}, \code{/}, \code{\%}, \code{+},
\code{-}, \code{\&}, \code{|}, \code{\^{}}, \code{=}, \code{*=},
\code{/=}, \code{\%=}, \code{+=}, \code{-=}, \code{\&=}, \code{|=},
\code{\^{}=}, all take two operands of the same, atomic, type, and
return a value of that type, that is, formally,
\[ \code{*}, \code{/}, \code{\%}, \code{+}, \code{-}, \code{\&},
\code{|}, \code{\^{}}, \code{=}, \code{*=}, \ldots, \code{\^{}=} 
\::\:
\forall \tau\in\{\code{bool}, \code{u8}, \code{u16}, \code{u32},
 \code{s8}, \code{s16},
 \code{s32}\}.\;\;\tau\times\tau\rightarrow\tau\,.\] 
The unary operators \code{+}, \code{-}, and \code{\~{}} all take one
 operand of any integer type and return a value of that type, that is,
 formally,
 \[\code{+}, \code{-}, \code{\~{}} \::\: 
 \forall \tau\in\{\code{bool}, \code{u8}, \code{u16}, \code{u32}, \code{s8},
 \code{s16}, \code{s32}\}.\;\;\tau\rightarrow\tau\,.\]
The binary operators \code{<}, \code{>}, \code{<=}, \code{>=},
 \code{==}, \code{!=},
\code{\&\&}, and \code{||} all take two operands of arbitrary atomic
types, and return a value of type \code{bool}, that is, formally
\[ \code{<}, \code{>}, \code{<=}, \code{>=},
   \code{==}, \code{!=},
   \code{\&\&}, \code{||} \::\:
   \forall \tau,\upsilon\in\{\code{bool}, \code{u8}, \code{u16}, \code{u32},
    \code{s8}, \code{s16},
    \code{s32}\}.\;\;\tau\times\upsilon\rightarrow\code{bool}\,.\] 
The unary operator \code{!} takes one operand of any integer types and
returns a value of type \code{bool}, that is, formally,
\[ \code{!} \::\:
   \forall \tau\in\{\code{bool}, \code{u8}, \code{u16}, \code{u32},
    \code{s8}, \code{s16},
    \code{s32}\}.\;\;\tau\rightarrow\code{bool}\,.\] For the binary
    operators \code{<<}, \code{>>}, \code{<<=}, and \code{>>=}, the
    value being shifted and the value indicating the number of bits to
    shift need not be of the same type, but both types must be atomic,
    and the entire expression gets the type of the value being
    shifted. Formally,
\[\code{<<}, \code{>>}, \code{<<=}, \code{>>=} \::\:
 \forall \sigma,\tau\in\{\code{bool}, \code{u8}, \code{u16}, \code{u32},
 \code{s8}, \code{s16},
 \code{s32}\}.\;\;\tau\times\sigma\rightarrow\tau\,.\]

A sequence \code{\{...\}} of expressions has the value and type of the
last expression in the sequence, or \code{void} if there are no
expressions. All expressions preceding the last are expected to have
type \code{void}, but as stated above, any other type will do when
\code{void} is expected.

An \code{if} expression consists of a condition, a consequent, and an
optional alternative. If the alternative is omitted, \code{unit} is
implied. The condition may be of any atomic type. The types of the
consequent and the alternative must match, but note again that because
any type can be used instead of \code{void}, if the type of either
branch is \code{void}, the other branch can have any type. The
alternative is evaluated only if the condition evaluates to \code{0};
otherwise, the consequent is evaluated. In either case, the entire
expression gets the value of the evaluated clause. To clarify:
\begin{verbatim}
fun frob () : void = {
    var x, y : u32;
    if 3 < 5 then do_something();  /* do_something() is executed */
    x = if 3 < 5 then 0 else 1;    /* x is set to 0 */
    x = if 3 > 5 then 0 else 1;    /* x is set to 1 */
    y = if 3 < 5 then 0;           /* This won't type check!! */
};
\end{verbatim}

There are two iteration constructs in Piffle. Both are different from
iteration constructs in comparable languages because of the need for
computing resource bounds. 

An expression of the form \[ \lit{for} \alpha \lit{in} \beta
   \grp{\lit{while} \gamma}\opt \lit{do} \delta \] has type
   \code{void}. The range $\beta$ must be of type \code{array $i$
   $\tau$}, for some integer $i$ and some atomic type $\tau$, and the
   variable $\alpha$ will be of type $\tau$ and scope over $\gamma$
   and $\delta$. The condition $\gamma$ may have any atomic type, and
   the body $\delta$ must be of type \code{void} (with any other type
   allowed).

An expression of the form \[ \lit{for} \alpha \lit{from} \beta_0
   \lit{to} \beta_1 \grp{\lit{while} \gamma}\opt \lit{do} \delta \]
   has type \code{void}. The bounds $\beta_{0,1}$ must be constant
   expressions of any atomic type, and $\alpha$ will be of type
   \code{u32} and scope over $\gamma$ and $\delta$. The condition
   $\gamma$ may have any atomic type, and the body $\delta$ must be of
   type \code{void} (with any other type allowed).

\section{Constant expressions}

Integer literals are assigned a special type (\code{sliteral}) by the
compiler. This type is considered atomic, and values of type
\code{sliteral} can always be used when any atomic type is expected. 


\section{Source files}

A source file consists of a number of declarations separated by
semicolons.

\clause{%
  \id{file} \is \grp{\id{declaration}\lit;}\star
}

\section{The filter function}

The function named \code{filter} has a special meaning.  It is used as
the root of the call graph (which is actually a tree, in the Piffle
case) by the compiler when checking for memory and CPU usage bounds.
It is also the function that gets called on every incoming packet.

Strictly speaking, this is dependent on which boilerplate is being
used. Boilerplates are a concept we will get to in the chapter about
the compiler, but rest assured that all currently available
boilerplates will cause \code{filter} to be called on each incoming
packet.


\section{Preprocessor}

It should be possible to run a Piffle program through a regular C
preprocessor (cpp). This is not done by default; you would have to do
it manually. Also make sure not to use nested comments if you want to
use cpp. The compiler does not interpret cpp line directives, so any
error messages you get about preprocessed code will refer to positions
in the preprocessed source.

\chapter{The Piffle compiler, pfc}

The Piffle compiler, pfc, is a fairly simple compiler, written in the
Haskell programming language, that outputs C code.  It is documented
in the pfc(1) man page, which is reproduced here in its
entirety.\footnote{The man pages included in this manual have been
converted to plain text using nroff and colcrt, which strips out most
of the formatting. If you know a better way of inserting a man page
into a \LaTeX{} document, please tell me.}

\section{The pfc(1) manual page}

\VerbatimInput[fontsize=\footnotesize]{pfc.txt}

\chapter{The pcap boilerplate, pcap.c}

As I explained in the pfc man page, you usually want to tell pfc to
embed a certain ``boilerplate'' into its output. The ``boilerplate''
is a file that typically contains a \verb.main(). function that
provides some means of gathering network packets, repeatedly calling
\verb.filter(). on them, and spitting them back out.

There is a second man page, pfc.pcap(1), that explains how to use a
Piffle program that has been complied using the pcap.c
boilerplate. Enjoy.

\section{The pfc.pcap(1) manual page}

\VerbatimInput[fontsize=\footnotesize]{pfc.pcap.txt}

\section{Writing programs for the pcap.c boilerplate}

The pcap.c boilerplate defines a function prototype \code{filter()} in
C as follows:

\begin{verbatim}
uint32_t filter(uint8_t inp[], uint32_t in_sz, uint8_t out[], uint32_t out_sz);
\end{verbatim}

This function is supposed to be declared in Piffle as follows:

\begin{verbatim}
fun filter( inp : u8[...], inp_sz : u32, 
            out : u8[...], out_sz : u32 ) : u32 = ...;
\end{verbatim}

The \code{inp} and \code{out} arrays are the input and output packets,
and the \code{inp\_sz} and \code{out\_sz} integers contain the size of
those arrays. The function is supposed to write things into the
\code{out} array, and return the actual length of the outgoing packet.

For instance, a \code{filter} function that simply copies the incoming
packet would look like this:

\begin{verbatim}
fun filter( inp : u8[65536], inp_sz : u32, 
            out : u8[65536], out_sz : u32 ) : u32 = {
     var i : u32;
     var out_sz_new : u32;

     out_sz_new = if inp_sz < out_sz then inp_sz else out_sz;

     for i from 0 to 65536 while i < out_sz_new do {
          out[i] = inp[i];
     };

     out_sz_new;
};
\end{verbatim}


\chapter{The test boilerplate, test.c}

There is another boilerplate called text.c, which exists mainly for
the benefit of the Piffle test suite, and is probably not useful for
anything else. It is nonetheless documented on its own manpage,
pfc.test(1).

\section{The pfc.test(1) manual page}

\VerbatimInput[fontsize=\footnotesize]{pfc.test.txt}

\chapter{An example of a packet filter}

\section{A packet filter in PFL}

\VerbatimInput[fontsize=\footnotesize]{pcap_tcp.pfl}

\newpage
\section{The packet filter translated to C by pfc}

\VerbatimInput[fontsize=\footnotesize]{pcap_tcp.c}


\end{document}
